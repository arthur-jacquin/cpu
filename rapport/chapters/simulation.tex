% ## Mise en pratique, validité de la solution
%
% > montrer les résultats de simulation
%
% Le testbench génère une horloge et un signal de réinitialisation actif pendant
% les premiers cycles d'horloge. Lors de la simulation, il suffit d'observer les
% valeurs de certains signaux du décodeur et des registres d'intérêt du RF.
%
% L'étude de l'évolution de NB_WAIT, index et ins semble confirmer la bonne
% gestion des sauts.
%
% L'éxécution du script stocke 0b10010000 = 144 dans le registre x1. Or
% fibo(12) = 144: on a bien le résultat attendu !
%
% l'ALU non complétèment testé, mais le coeur des opérations et fait automatiquement,
% c'est surtout la partie contrôle de l'ALU qui est réalisé dans le projet et testé
% avec le script
