\chapter{Architecture}
\label{ch:architecure}

\section{Conséquences du pipeline, gestion des aléas}
\label{sec:pipeline}

Le choix d'une architecture pipelinée\cite{pipeline} a de nombreuses
conséquences. Premièrement, le fait d'avoir plusieurs étapes de traitement
signifie que les différentes parties du processeur n'ont pas besoin d'être
synchrones (le synchronisme étant géré par les registres tampons séparant les
étapes), ce qui change la syntaxe et les possibilités en VHDL. Deuxièmement,
cela crée des \textit{aléas} :
\begin{itemize}
\item de contrôle (incertitude de l'instruction à charger lors d'un saut
    conditionnel),
\item de données (utilisation d'un registre dont la valeur a été modifiée par
    une instruction précédente depuis sa lecture dans le
    \textit{register file}). \\
\end{itemize}

Plusieurs stratégies peuvent être utilisées pour contrer ces aléas. Une méthode
est d'introduire des pauses pour attendre que la donnée importante (adresse de
l'instruction à charger, valeur du registre...) soit connue. Pour des raisons de
simplicité, c'est cette méthode qu'utilise notre processeur pour gérer les aléas
de contrôle. Lorsqu'une instruction est un saut, le décodeur s'occupe de
remplacer le traitement des instructions suivantes par celui d'une instruction
sans effet (ici, \texttt{or x0, x0, x0}), et ce jusqu'à ce que l'adresse après
le saut soit déterminée. \\

Pour ce qui est des aléas de données, un ouvrage\cite{riscv} propose une méthode
systématique pour les contrer : le \textit{forwarding}/\textit{bypassing}. Dans
la même idée, nous avons introduit un cache dans l'étape de calcul mémorisant
l'adresse et la valeur du dernier registre calculé. Ainsi, si le calcul utilise
ce registre, la valeur dans le cache est utilisée au lieu de celle provenant du
\textit{register file}.

\section{Architecture générale}

L'architecture générale est illustrée en figure \ref{fig:architecture}. Les
barres bleues symbolisent les registres tampons entre les étapes. Pour plus de
clarté, certains détails sont omis :
\begin{itemize}
\item connections à l'arbre d'horloge et au signal de réinitialisation,
\item tailles des bus,
\item opérations occasionnelles de redimensionnement des bus.
\end{itemize}

\begin{landscape}
\begin{figure}[!h]
    \centering
    \chapter{Architecture}
\label{ch:architecure}

\section{Conséquences du pipeline, gestion des aléas}
\label{sec:pipeline}

Le choix d'une architecture pipelinée\cite{pipeline} a de nombreuses
conséquences. Premièrement, le fait d'avoir plusieurs étapes de traitement
signifie que les différentes parties du processeur n'ont pas besoin d'être
synchrones (le synchronisme étant géré par les registres tampons séparant les
étapes), ce qui change la syntaxe et les possibilités en VHDL. Deuxièmement,
cela crée des \textit{aléas} :
\begin{itemize}
\item de contrôle (incertitude de l'instruction à charger lors d'un saut
    conditionnel),
\item de données (utilisation d'un registre dont la valeur a été modifiée par
    une instruction précédente depuis sa lecture dans le
    \textit{register file}). \\
\end{itemize}

Plusieurs stratégies peuvent être utilisées pour contrer ces aléas. Une méthode
est d'introduire des pauses pour attendre que la donnée importante (adresse de
l'instruction à charger, valeur du registre...) soit connue. Pour des raisons de
simplicité, c'est cette méthode qu'utilise notre processeur pour gérer les aléas
de contrôle. Lorsqu'une instruction est un saut, le décodeur s'occupe de
remplacer le traitement des instructions suivantes par celui d'une instruction
sans effet (ici, \texttt{or x0, x0, x0}), et ce jusqu'à ce que l'adresse après
le saut soit déterminée. \\

Pour ce qui est des aléas de données, un ouvrage\cite{riscv} propose une méthode
systématique pour les contrer : le \textit{forwarding}/\textit{bypassing}. Dans
la même idée, nous avons introduit un cache dans l'étape de calcul mémorisant
l'adresse et la valeur du dernier registre calculé. Ainsi, si le calcul utilise
ce registre, la valeur dans le cache est utilisée au lieu de celle provenant du
\textit{register file}.

\section{Architecture générale}

L'architecture générale est illustrée en figure \ref{fig:architecture}. Les
barres bleues symbolisent les registres tampons entre les étapes. Pour plus de
clarté, certains détails sont omis :
\begin{itemize}
\item connections à l'arbre d'horloge et au signal de réinitialisation,
\item tailles des bus,
\item opérations occasionnelles de redimensionnement des bus.
\end{itemize}

\begin{landscape}
\begin{figure}[!h]
    \centering
    \chapter{Architecture}
\label{ch:architecure}

\section{Conséquences du pipeline, gestion des aléas}
\label{sec:pipeline}

Le choix d'une architecture pipelinée\cite{pipeline} a de nombreuses
conséquences. Premièrement, le fait d'avoir plusieurs étapes de traitement
signifie que les différentes parties du processeur n'ont pas besoin d'être
synchrones (le synchronisme étant géré par les registres tampons séparant les
étapes), ce qui change la syntaxe et les possibilités en VHDL. Deuxièmement,
cela crée des \textit{aléas} :
\begin{itemize}
\item de contrôle (incertitude de l'instruction à charger lors d'un saut
    conditionnel),
\item de données (utilisation d'un registre dont la valeur a été modifiée par
    une instruction précédente depuis sa lecture dans le
    \textit{register file}). \\
\end{itemize}

Plusieurs stratégies peuvent être utilisées pour contrer ces aléas. Une méthode
est d'introduire des pauses pour attendre que la donnée importante (adresse de
l'instruction à charger, valeur du registre...) soit connue. Pour des raisons de
simplicité, c'est cette méthode qu'utilise notre processeur pour gérer les aléas
de contrôle. Lorsqu'une instruction est un saut, le décodeur s'occupe de
remplacer le traitement des instructions suivantes par celui d'une instruction
sans effet (ici, \texttt{or x0, x0, x0}), et ce jusqu'à ce que l'adresse après
le saut soit déterminée. \\

Pour ce qui est des aléas de données, un ouvrage\cite{riscv} propose une méthode
systématique pour les contrer : le \textit{forwarding}/\textit{bypassing}. Dans
la même idée, nous avons introduit un cache dans l'étape de calcul mémorisant
l'adresse et la valeur du dernier registre calculé. Ainsi, si le calcul utilise
ce registre, la valeur dans le cache est utilisée au lieu de celle provenant du
\textit{register file}.

\section{Architecture générale}

L'architecture générale est illustrée en figure \ref{fig:architecture}. Les
barres bleues symbolisent les registres tampons entre les étapes. Pour plus de
clarté, certains détails sont omis :
\begin{itemize}
\item connections à l'arbre d'horloge et au signal de réinitialisation,
\item tailles des bus,
\item opérations occasionnelles de redimensionnement des bus.
\end{itemize}

\begin{landscape}
\begin{figure}[!h]
    \centering
    \chapter{Architecture}
\label{ch:architecure}

\section{Conséquences du pipeline, gestion des aléas}
\label{sec:pipeline}

Le choix d'une architecture pipelinée\cite{pipeline} a de nombreuses
conséquences. Premièrement, le fait d'avoir plusieurs étapes de traitement
signifie que les différentes parties du processeur n'ont pas besoin d'être
synchrones (le synchronisme étant géré par les registres tampons séparant les
étapes), ce qui change la syntaxe et les possibilités en VHDL. Deuxièmement,
cela crée des \textit{aléas} :
\begin{itemize}
\item de contrôle (incertitude de l'instruction à charger lors d'un saut
    conditionnel),
\item de données (utilisation d'un registre dont la valeur a été modifiée par
    une instruction précédente depuis sa lecture dans le
    \textit{register file}). \\
\end{itemize}

Plusieurs stratégies peuvent être utilisées pour contrer ces aléas. Une méthode
est d'introduire des pauses pour attendre que la donnée importante (adresse de
l'instruction à charger, valeur du registre...) soit connue. Pour des raisons de
simplicité, c'est cette méthode qu'utilise notre processeur pour gérer les aléas
de contrôle. Lorsqu'une instruction est un saut, le décodeur s'occupe de
remplacer le traitement des instructions suivantes par celui d'une instruction
sans effet (ici, \texttt{or x0, x0, x0}), et ce jusqu'à ce que l'adresse après
le saut soit déterminée. \\

Pour ce qui est des aléas de données, un ouvrage\cite{riscv} propose une méthode
systématique pour les contrer : le \textit{forwarding}/\textit{bypassing}. Dans
la même idée, nous avons introduit un cache dans l'étape de calcul mémorisant
l'adresse et la valeur du dernier registre calculé. Ainsi, si le calcul utilise
ce registre, la valeur dans le cache est utilisée au lieu de celle provenant du
\textit{register file}.

\section{Architecture générale}

L'architecture générale est illustrée en figure \ref{fig:architecture}. Les
barres bleues symbolisent les registres tampons entre les étapes. Pour plus de
clarté, certains détails sont omis :
\begin{itemize}
\item connections à l'arbre d'horloge et au signal de réinitialisation,
\item tailles des bus,
\item opérations occasionnelles de redimensionnement des bus.
\end{itemize}

\begin{landscape}
\begin{figure}[!h]
    \centering
    \input{appendices/architecture.tex}
    \caption{Architecture générale du processeur}
    \label{fig:architecture}
\end{figure}
\end{landscape}

\section{Implémentation en VHDL}

Comme expliqué en partie \ref{sec:pipeline}, les composants sont majoritairement
décrits de façon asynchrone. L'élément le plus complexe et le plus intéressant
est le décodeur, décrit dans \texttt{decoder.vhd} (\ref{decoder}). \\

\lstinputlisting[language=VHDL, caption=\texttt{decoder.vhd}, label=decoder,
    basicstyle=\ttfamily\footnotesize]{../vhdl/decoder.vhd}

On peut remarquer l'utilisation du signal \texttt{NB\_WAIT}, qui permet gérer
les sauts. Il détermine en effet le nombre de cycles d'horloge pendant lesquels
l'instruction lue dans la mémoire sera ignorée et remplacée par une instruction
sans effet. Ce signal est modifié par un processus synchrone. \\

Le décodeur est également le seul composant du processeur dans lequel est pris
en compte le signal de réinitialisation. Lorsque ce signal est actif, le seul
effet est de traiter le saut inconditionnel vers l'instruction d'addresse
\texttt{0}. Cela signifie que les programmes ne peuvent pas faire de
suppositions sur la valeur des registres ou de la pile : ils doivent initialiser
l'ensemble des données qu'ils utilisent.

    \caption{Architecture générale du processeur}
    \label{fig:architecture}
\end{figure}
\end{landscape}

\section{Implémentation en VHDL}

Comme expliqué en partie \ref{sec:pipeline}, les composants sont majoritairement
décrits de façon asynchrone. L'élément le plus complexe et le plus intéressant
est le décodeur, décrit dans \texttt{decoder.vhd} (\ref{decoder}). \\

\lstinputlisting[language=VHDL, caption=\texttt{decoder.vhd}, label=decoder,
    basicstyle=\ttfamily\footnotesize]{../vhdl/decoder.vhd}

On peut remarquer l'utilisation du signal \texttt{NB\_WAIT}, qui permet gérer
les sauts. Il détermine en effet le nombre de cycles d'horloge pendant lesquels
l'instruction lue dans la mémoire sera ignorée et remplacée par une instruction
sans effet. Ce signal est modifié par un processus synchrone. \\

Le décodeur est également le seul composant du processeur dans lequel est pris
en compte le signal de réinitialisation. Lorsque ce signal est actif, le seul
effet est de traiter le saut inconditionnel vers l'instruction d'addresse
\texttt{0}. Cela signifie que les programmes ne peuvent pas faire de
suppositions sur la valeur des registres ou de la pile : ils doivent initialiser
l'ensemble des données qu'ils utilisent.

    \caption{Architecture générale du processeur}
    \label{fig:architecture}
\end{figure}
\end{landscape}

\section{Implémentation en VHDL}

Comme expliqué en partie \ref{sec:pipeline}, les composants sont majoritairement
décrits de façon asynchrone. L'élément le plus complexe et le plus intéressant
est le décodeur, décrit dans \texttt{decoder.vhd} (\ref{decoder}). \\

\lstinputlisting[language=VHDL, caption=\texttt{decoder.vhd}, label=decoder,
    basicstyle=\ttfamily\footnotesize]{../vhdl/decoder.vhd}

On peut remarquer l'utilisation du signal \texttt{NB\_WAIT}, qui permet gérer
les sauts. Il détermine en effet le nombre de cycles d'horloge pendant lesquels
l'instruction lue dans la mémoire sera ignorée et remplacée par une instruction
sans effet. Ce signal est modifié par un processus synchrone. \\

Le décodeur est également le seul composant du processeur dans lequel est pris
en compte le signal de réinitialisation. Lorsque ce signal est actif, le seul
effet est de traiter le saut inconditionnel vers l'instruction d'addresse
\texttt{0}. Cela signifie que les programmes ne peuvent pas faire de
suppositions sur la valeur des registres ou de la pile : ils doivent initialiser
l'ensemble des données qu'ils utilisent.

    \caption{Architecture générale du processeur}
    \label{fig:architecture}
\end{figure}
\end{landscape}

\section{Implémentation en VHDL}

Comme expliqué en partie \ref{sec:pipeline}, les composants sont majoritairement
décrits de façon asynchrone. L'élément le plus complexe et le plus intéressant
est le décodeur, décrit dans \texttt{decoder.vhd} (\ref{decoder}). \\

\lstinputlisting[language=VHDL, caption=\texttt{decoder.vhd}, label=decoder,
    basicstyle=\ttfamily\footnotesize]{../vhdl/decoder.vhd}

On peut remarquer l'utilisation du signal \texttt{NB\_WAIT}, qui permet de gérer
les sauts. Il détermine en effet le nombre de cycles d'horloge pendant lesquels
l'instruction lue dans la mémoire sera ignorée et remplacée par une instruction
sans effet. Ce signal est modifié par un processus synchrone. \\

Le décodeur est également le seul composant du processeur dans lequel est pris
en compte le signal de réinitialisation. Lorsque ce signal est actif, le seul
effet est de traiter le saut inconditionnel vers l'instruction d'addresse
\texttt{0}. Cela signifie que les programmes ne peuvent pas faire de
suppositions sur la valeur des registres ou de la pile : ils doivent initialiser
l'ensemble des données qu'ils utilisent.
