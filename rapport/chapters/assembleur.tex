\chapter{Assembleur}
\label{ch:assembleur}

\section{Motivation}

Le code binaire, c'est-à-dire des suites d'instructions encodé comme précisé
dans le chapitre \ref{ch:isa}, est très peu lisible et manipulable tel quel.
C'est pour faciliter l'écriture et la lecture de programme par l'humain que des
mnémoniques sont associées à chaque instruction. \\

Pour aller plus loin dans le projet, nous nous sommes essayés à la conception
d'un assembleur, programme qui transforme un programme composé de mnémoniques
(sous forme de fichier textuel facile à manipuler) en code binaire
intelligible pour notre processeur. Le code source est disponible en annexe
\ref{ch:asb}.

\section{Fonctionnement}

Pour des raisons de simplicité, l'assembleur est plutôt simple, et certaines
hypothèses ont été faites (fichier ASCII uniquement, longueur maximale de
ligne, immédiats positifs...). La syntaxe choisie imite fortement celles
d'autres langages assembleurs. \\

Notre assembleur opère en deux phases :
\begin{enumerate}
\item un premier parcours du programme réalise une analyse syntaxique
    (reconnaissance des mnémoniques et des arguments, ignorance des
    commentaires, formation de la liste des labels) et prépare la génération de
    code binaire,
\item un deuxième passage permet de renseigner l'adresse des labels utilisés
    (étape de \textit{link}). \\
\end{enumerate}

Pour éviter des problèmes à l'éxecution, certaines vérifications sont faites
par l'assembleur :
\begin{itemize}
\item validité de la syntaxe,
\item existence et unicité de chaque label utilisé,
\item taille et positivité des immédiats.
\end{itemize}
En cas d'échec, des messages d'erreur incluant le numéro de la ligne
problématique facilitent le déboguage du programme.

\section{Exemple de programme}

\lstinputlisting[caption=\texttt{fibo.asb}, label=fibo]{../test.asb}

Pour tester le processeur, on utilise le programme \ref{fibo} qui calcule
$F_{12}$ (le treizième terme de la suite de Fibonacci\cite{fibo}), car il permet
d'utiliser plusieurs fonctionnalités :
\begin{itemize}
\item copie de valeurs entre plusieurs registres,
\item quelques opérations arithmétiques,
\item utilisation d'immédiats et de registres pour les opérandes,
\item sauts conditionnels et inconditionnels. \\
\end{itemize}

Pour assembler ce programme, il faut compiler l'assembleur (\texttt{asb.c}) puis
exécuter l'assembleur en spécifiant le nom du programme à assembler (commandes
\ref{assemble}).

\begin{lstlisting}[caption=Assemblage de \texttt{fibo.asb}, label=assemble]
cc -o asb asb.c
./asb fibo.asb > fibo.out
\end{lstlisting}

Le résultat est ici enregistré dans \texttt{fibo.out} (\ref{out}).

\lstinputlisting[caption=\texttt{fibo.out}, label=out]{../out}
