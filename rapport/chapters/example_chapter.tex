\chapter{Example chapter}
\label{ch:example}

\section{Labels and references}

\subsection{Subsection}
\label{subsec:example}

% TODO: label, ref, pageref for floats, cite

Let's explore how to make references. You can reference a bibliography entry
(see \cite{einstein}), and even several at once (see \cite{dirac, knuthwebsite}).

It is also possible to reference chapters, sections and subsections, for example
this paragraph is in chapter \ref{ch:example}, and subsection \ref{subsec:example}.
It is also possible to reference chapters, sections and subsections, for example
this paragraph is in chapter \ref{ch:example}, and subsection \ref{subsec:example}.
It is also possible to reference chapters, sections and subsections, for example
this paragraph is in chapter \ref{ch:example}, and subsection \ref{subsec:example}.

Let's explore how to make references. You can reference a bibliography entry
(see \cite{einstein}), and even several at once (see \cite{dirac, knuthwebsite}).

It is also possible to reference chapters, sections and subsections, for example
this paragraph is in chapter \ref{ch:example}, and subsection \ref{subsec:example}.
It is also possible to reference chapters, sections and subsections, for example
this paragraph is in chapter \ref{ch:example}, and subsection \ref{subsec:example}.
It is also possible to reference chapters, sections and subsections, for example
this paragraph is in chapter \ref{ch:example}, and subsection \ref{subsec:example}.


\subsection{Another subsection}


\section{Elements}

\subsection{Chapters, sections, subsections}

Maybe I should say a word about this.

\subsection{Figure}

\begin{figure}[h]
    \centering
    \includegraphics[width=0.25\textwidth]{figures/entity_logo_1.png}
    \caption{A nice logo}
    \label{fig:cs_logo}
\end{figure}

All references to the figure \ref{fig:cs_logo}, appearing on page
\pageref{fig:cs_logo}, must use the \texttt{ref} and \texttt{pageref} commands.

\subsection{Table}

%\begin{table}[ht]
%    \centering
%    \begin{tabular}{lcc}
%    \toprule
%    Person & Treatment A & Treatment B \\
%    \midrule
%    John Smith & 1 & 2 \\
%    Jane Doe & -- & 3 \\
%    Mary Johnson & 4 & 5 \\
%    \bottomrule
%    \end{tabular}
%    \caption{A professional looking table}
%    \label{tab:treatments}
%\end{table}

All references to the table \ref{tab:treatments}, appearing on page
\pageref{tab:treatments}, must use the \texttt{ref} and \texttt{pageref} commands.
Let's explore how to make references. You can reference a bibliography entry
(see \cite{einstein}), and even several at once (see \cite{dirac, knuthwebsite}).

Let's explore how to make references. You can reference a bibliography entry
(see \cite{einstein}), and even several at once (see \cite{dirac, knuthwebsite}).
Let's explore how to make references. You can reference a bibliography entry
(see \cite{einstein}), and even several at once (see \cite{dirac, knuthwebsite}).


\subsection{Equation}

\begin{equation}
    e^{i \pi} + 1 = 0
    \label{eq:identity}
\end{equation}

The equation \ref{eq:identity} is remarquable.

\subsection{Script}

If you just need some inline code, you can use this command:
\code{print("Hello World!")}. For a script, you can use this environment:

\begin{lstlisting}[language=Python, caption=A fast exponentation]
def expo(x, n):
    if n == 0:
        return 1
    elif n == 1:
        return x
    elif n % 2:
        return x * expo(x * x, n // 2)
    else:
        return expo(x * x, n // 2)
\end{lstlisting}

It is also possible to load an external script, in its entirety or just a range
of lines:

\lstinputlisting[language=Python, caption=Recursive PGCD]{scripts/pgcd.py}

\lstinputlisting[language=Python, firstline=1, lastline=1,
    caption=PGCD prototype, label=code:essai]{scripts/pgcd.py}

This is code \ref{code:essai}.
