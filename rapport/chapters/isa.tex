\chapter{Conception de l'ISA (Instruction Set Architecture)}

\section{Tailles critiques}

La première étape est de choisir le nombre et la taille des registres. C'est en
effet crucial pour la conception de l'ISA :
\begin{itemize}
\item le nombre de registres détermine le nombre de bits nécessaires pour une
    adresse de registre,
\item pour assigner une valeur arbitraire à un registre en une instruction, il
    faut qu'au moins un format accueille un immédiat de la même taille que les
    registres. \\
\end{itemize}

Pour obtenir un CPU raisonnablement capable, nous avons choisi 16 registres de
16 bits.

\section{Formats}

Alors que la majorité des instructions nécessite de petits opérandes (comme des
addresses de registres ou des petits immédiats), quelques unes utilisent un
unique immédiat très grand (de la même taille que les registres). \\

Par souci de compacité et pour garder une taille d'instruction relativement
faible, l'\textit{opcode} (le code de l'instruction) est encodé en deux parties :
\begin{itemize}
\item le champ \texttt{op}, commun à tous les formats,
\item le champ \texttt{funct}, permettant une plus grande granularité avec les
    instructions manipulant plusieurs opérandes mais inexistant pour celles ou
    un immédiat occupe beaucoup de bits.
\end{itemize}

\begin{table}[ht]
    \centering
    \begin{tabular}{cl}
    \toprule
    \texttt{fmt} & Champs (nom et taille) \\
    \midrule
    \texttt{00} & \texttt{fmt[2] op[4] dest[4] src1[4] funct[2] [6] src2[4]} \\
    \texttt{01} & \texttt{fmt[2] op[4] dest[4] src1[4] funct[2] imm[10]} \\
    \texttt{10} & \texttt{fmt[2] op[4] dest[4] imm[16]} \\
    \bottomrule
    \end{tabular}
    \caption{Formats de l'ISA}
    \label{tab:formats}
\end{table}

Il y a donc plusieurs catégories d'instructions (table \ref{tab:instructions_types}).

\begin{table}[ht]
    \centering
    \begin{tabular}{cc}
    \toprule
    Catégorie & Description \\
    \midrule
    A & nécessite \texttt{funct}, \texttt{fmt} = \texttt{00} ou \texttt{01} \\
    B & nécessite un grand immédiat, \texttt{fmt} = \texttt{10} \\
    C & autres, \texttt{fmt} = \texttt{00} ou \texttt{10} \\
    \bottomrule
    \end{tabular}
    \caption{Catégories d'instructions}
    \label{tab:instructions_types}
\end{table}

\section{Instructions et mnémoniques réservées}

Les codes d'instructions du tableau \ref{tab:opcodes} sont réservés et ne
pourront plus être modifié. En revanche, l'ajout de fonctionnalités est possible
avec les codes n'y figurant pas.

\begin{table}[ht]
    \centering
    \begin{tabular}{ccccl}
    \toprule
    \texttt{op} & Catégorie & \texttt{funct} & Mnémonique & Commentaire \\
    \midrule
    \texttt{0000} & A & & & Opérations bit à bit \\
    & & \texttt{00} & \texttt{not} & \\
    & & \texttt{01} & \texttt{and} & \\
    & & \texttt{10} & \texttt{or} & \\
    & & \texttt{11} & \texttt{xor} & \\
    \texttt{0001} & A & & & Décalages de bits \\
    & & \texttt{00} & \texttt{rls} & \\
    & & \texttt{01} & \texttt{lls} & \\
    \texttt{0010} & A & & & Opérations arithmétiques \\
    & & \texttt{00} & \texttt{add} & \\
    & & \texttt{01} & \texttt{sub} & \\
    & & \texttt{10} & \texttt{mul} & \\
    \texttt{0011} & A & & & Manipulation de la pile \\
    & & \texttt{00} & \texttt{pop} & \\
    & & \texttt{01} & \texttt{read} & \\
    & & \texttt{10} & \texttt{push} & \\
    \texttt{0100} & B & & \texttt{jmp} & \\
    \texttt{0101} & C & & \texttt{mov} & \\
    \texttt{0110} & B & & \texttt{call} & \\
    \texttt{0111} & C & & \texttt{ret} & \\
    \bottomrule
    \end{tabular}
    \caption{Codes des instructions}
    \label{tab:opcodes}
\end{table}

Certaines \textit{pseudo-instructions} (mnémoniques se rattachant à une
instruction existante) sont comprises par l'assembleur (voire chapitre
\ref{ch:assembleur}) :
\begin{itemize}
\item \texttt{inc}, \texttt{dec}
\item \texttt{beq}, \texttt{bne}, \texttt{bgt}, \texttt{bge}, \texttt{blt}, \texttt{ble}
\end{itemize}

\section{Sauts (in)conditionnels}

Par souci de compacité et pour ne pas occuper inutilement un grand nombre de
codes d'instructions, les sauts conditionnels et inconditionnels sont encodés
avec le même code (\texttt{op = 0100}). Plus précisément, tous les sauts sont
considérés conditionnels, avec une condition tautologique pour les sauts
"inconditionnels". \\

Les conditions sont encodés avec un masque\cite{bit_mask} dans le champ
\texttt{dest}. Ce masque peut être vu comme un tableau de drapeaux d'exigences.
Par exemple, le masque \texttt{0000} n'a aucun drapeau actif, donc aucune
exigence, et est utilisé pour pour les sauts inconditionnels. En revanche, le
masque \texttt{0110} exige les comparaisons "non égaux" et "plus grand que", ce
qui correspond à "strictement plus grand" et utilisé pour le saut conditionnel
associé à la pseudo-instruction \texttt{bgt} (pour \textit{branch greater
than}). \\

Grâce à l'utilisation des pseudo-instructions, les masques sont gérés par
l'assembleur et non l'utilisateur.

\section{\texttt{read}}

L'instruction associée à la mnémonique \texttt{read} (\texttt{op = 0011},
\texttt{funct = 01}) permet de lire une valeur dans la pile à une profondeur
quelconque (spécifiée avec \texttt{imm} ou stockée dans le registre désigné par
\texttt{src2}), sans modifier la pile. C'est utile pour les appels de
fonctions : on pousse les paramètres dans la pile, et on les lit quand
nécessaire (lors de la compilation, l'état de la pile et donc la
profondeur à laquelle se trouve chaque paramètre sont connus).
