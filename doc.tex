\documentclass[border=10]{standalone}

% https://mirrors.ircam.fr/pub/CTAN/graphics/pgf/contrib/circuitikz/doc/circuitikzmanual.pdf
\usepackage{circuitikz}

\begin{document}

\begin{circuitikz}
\ctikzset{
    logic ports=ieee,
    logic ports/scale=0.7,
    multipoles/dipchip/width=2,
    multipoles/dipchip/pin spacing=0.2,
    no topmark,
    external pins width=0,
    hide numbers,
    font={\ttfamily\tiny},
    muxdemuxes/fill=lightgray,
    radius=0.07,
}
\tikzset{
    mux/.style={
        muxdemux,
        muxdemux def={Lh=1, Rh=0.8, w=0.2, NT=1, NL=2, NB=0, NR=1},
    },
    stage/.style={
        dipchip,
        circuitikz/multipoles/dipchip/width=0.2,
        fill=cyan,
    },
}

\pgfmathsetmacro{\mp}{0.1} % minimal spacing between wires
\pgfmathsetmacro{\ps}{0.2/0.7} % pin spacing
\pgfmathsetmacro{\namesspacing}{1}
\pgfmathsetmacro{\padding}{0.5}
\pgfmathtruncatemacro{\cachedepth}{2}
\newcommand{\namescolor}{darkgray}

% ROM
\draw (0, 0)
    node[dipchip, num pins=6] (ROM) {}
    node[below] at (ROM.n) {\normalsize ROM}
    node[right] at (ROM.bpin 3) {index}
    node[left] at (ROM.bpin 4) {instruction}
;

% ST1
\draw (ROM.bpin 6) ++(\padding, 0)
    node[stage, anchor=bpin 5, num pins=94] (ST1) {}
    node[below] at (ST1.n) {\normalsize 1}
    node[\namescolor] at ([shift={(0, \namesspacing)}] ROM.n |- ST1.n) {\normalsize INSTRUCTION FETCH}
;
\draw
    (ROM.bpin 3) -| ([shift={(-\ps, 0)}] ROM.bpin 3 |- ST1.bpin 3) node[circ] (index1) {}
;

% DECODER
\draw (ST1.bpin 94) ++(\padding, 0)
    node[dipchip, anchor=bpin 1, num pins=94] (DECODER) {}
    node[below] at (DECODER.n) {\normalsize DECODER}
    node[right] at (DECODER.bpin 3) {index}
    node[right] at (DECODER.bpin 7) {instruction}

    % for INDEX MANAGER
    node[left] at (DECODER.bpin 91) {jump}
    node[left] at (DECODER.bpin 90) {conditions}
    node[left] at (DECODER.bpin 89) {next\_index}
    node[left] at (DECODER.bpin 88) {use\_RAM\_index}
    node[left] at (DECODER.bpin 87) {jump\_index}

    % for MEMORY
    node[left] at (DECODER.bpin 85) {use\_reg\_imm}
    node[left] at (DECODER.bpin 84) {enable\_RAM}
    node[left] at (DECODER.bpin 83) {in\_index\_RAM}
    node[left] at (DECODER.bpin 82) {read\_index\_RAM}

    % for ALU
    node[left] at (DECODER.bpin 77) {operation}
    node[left] at (DECODER.bpin 76) {use\_src2}
    node[left] at (DECODER.bpin 75) {immediate}

    % for REGISTER FILE, SELECTOR
    node[left] at (DECODER.bpin 72) {src1}
    node[left] at (DECODER.bpin 70) {src2}

    % for CACHE
    node[left] at (DECODER.bpin 53) {valid}
    node[left] at (DECODER.bpin 52) {dest}
    node[left] at (DECODER.bpin 51) {use\_ALU}
;
\draw
    (DECODER.bpin 3) -- (index1)
    (DECODER.bpin 7) -- (ROM.bpin 4)
;

% ST2
\draw (DECODER.bpin 94) ++(\padding,0)
    node[stage, anchor=bpin 1, num pins=94] (ST2) {}
    node[below] at (ST2.n) {\normalsize 2}
    node[\namescolor] at ([shift={(0, \namesspacing)}] $(ST1.n)!0.5!(ST2.n)$) {\normalsize DECODE}
;

% REGISTER FILE
\draw (ST2.bpin 66) ++(\padding + 2*\ps, 0)
    node[dipchip, anchor=bpin 1, num pins=14] (RF) {}
    node[below] at (RF.n) {\normalsize REGISTER FILE}
    node[right] at (RF.bpin 3) {src1}
    node[right] at (RF.bpin 4) {src2}
    node[right] at (RF.bpin 5) {enable}
    node[right] at (RF.bpin 6) {dest}
    node[right] at (RF.bpin 7) {in}
    node[left] at (RF.bpin 12) {out1}
    node[left] at (RF.bpin 11) {out2}
;
\draw
    (RF.bpin 3) -| ([shift={(-1*\ps, 0)}] RF.bpin 3 |- DECODER.bpin 72) node[circ] (src1) {} -- (DECODER.bpin 72)
    (RF.bpin 4) -| ([shift={(-2*\ps, 0)}] RF.bpin 4 |- DECODER.bpin 70) node[circ] (src2) {} -- (DECODER.bpin 70)
;

% ST3
\draw (RF.bpin 14) ++(\padding,0)
    node[stage, anchor=bpin 29, num pins=94] (ST3) {}
    node[below] at (ST3.n) {\normalsize 3}
    node[\namescolor] at ([shift={(0, \namesspacing)}] $(ST2.n)!0.5!(ST3.n)$) {\normalsize REGISTER FETCH}
;

% RAM
\draw (ST3.bpin 86) ++(\padding+3*\ps+2*\mp, 0)
    node[dipchip, anchor=bpin 1, num pins=14] (RAM) {}
    node[below] at (RAM.n) {\normalsize MEMORY}
    node[right] at (RAM.bpin 3) {enable}
    node[right] at (RAM.bpin 4) {in\_index}
    node[right] at (RAM.bpin 5) {out\_index}
    node[right] at (RAM.bpin 7) {in}
    node[left] at (RAM.bpin 12) {out}
;
\draw (RAM.bpin 7) -- ++(-\ps, 0)
    node[mux, anchor=brpin 1] (muxRAM) {};
\draw
    (RAM.bpin 3) -- (DECODER.bpin 84)
    (RAM.bpin 4) -- (DECODER.bpin 83)
    (RAM.bpin 5) -- (DECODER.bpin 82)
    (muxRAM.btpin 1) |- (DECODER.bpin 85)
    (muxRAM.blpin 1) -| ([shift={(-\ps, 0)}] muxRAM.blpin 1 |- DECODER.bpin 89) node[circ] (nextindex) {} -- (DECODER.bpin 89)
;

% SEL1
\draw (ST3.bpin 74 -| RAM.bpin 1)
    node[dipchip, anchor=bpin 1, num pins=14,
        circuitikz/multipoles/dipchip/width=1.3] (SEL1) {}
    node[below] at (SEL1.n) {\normalsize SELECTOR}
    node[right] at (SEL1.bpin 3) {src}
    node[right] at (SEL1.bpin 4) {in}
    node[right] at (SEL1.bpin 5) {valid\_cache[\cachedepth]}
    node[right] at (SEL1.bpin 6) {addr\_cache[\cachedepth]}
    node[right] at (SEL1.bpin 7) {cache[\cachedepth]}
    node[left] at (SEL1.bpin 12) {out}
;
\draw
    (SEL1.bpin 3) -- (src1)
    (SEL1.bpin 4) -- ++(-3*\ps-2*\mp, 0) |- (RF.bpin 12)
;

% SEL2
\draw (ST3.bpin 66 -| SEL1.bpin 1)
    node[dipchip, anchor=bpin 1, num pins=14,
        circuitikz/multipoles/dipchip/width=1.3] (SEL2) {}
    node[below] at (SEL2.n) {\normalsize SELECTOR}
    node[right] at (SEL2.bpin 3) {src}
    node[right] at (SEL2.bpin 4) {in}
    node[right] at (SEL2.bpin 5) {valid\_cache[\cachedepth]}
    node[right] at (SEL2.bpin 6) {addr\_cache[\cachedepth]}
    node[right] at (SEL2.bpin 7) {cache[\cachedepth]}
    node[left] at (SEL2.bpin 12) {out}
;
\draw
    (SEL2.bpin 3) -- ++(-2*\ps-2*\mp, 0) |- (src2)
    (SEL2.bpin 4) -- (RF.bpin 11)
;

% ALU
\draw (SEL1.bpin 12) -- ++(4*\ps + 0.2, 0)
    node[ALU, anchor=blpin 1] (ALU) {\rotatebox{90}{\normalsize ALU}};
\draw (ALU.blpin 2) -- ++(-\ps, 0) node[circ] (src_imm) {} -- ++(-\ps, 0)
    node[mux, anchor=brpin 1] (muxALU) {};
\draw
    (src_imm) |- ([shift={(-\ps, \ps)}] muxRAM.blpin 2 |- DECODER.bpin 77) |- (muxRAM.blpin 2)
    (ALU.btpin 1) |- (DECODER.bpin 77)
    (muxALU.btpin 1) |- (DECODER.bpin 76)
    (muxALU.blpin 1) -- ++(-\ps, 0) |- (DECODER.bpin 75)
    (muxALU.blpin 2) -- ++(-\ps, 0) |- (SEL2.bpin 12)
;

% CACHE
\draw (ST3.bpin 55 -| RAM.bpin 1)
    node[dipchip, anchor=bpin 1, num pins=16] (CACHE) {}
    node[below] at (CACHE.n) {\normalsize CACHE}
    node[right] at (CACHE.bpin 3) {valid\_in}
    node[right] at (CACHE.bpin 4) {addr\_in}
    node[right] at (CACHE.bpin 7) {in}
    node[left] at (CACHE.bpin 14) {valid\_cache[\cachedepth]}
    node[left] at (CACHE.bpin 13) {addr\_cache[\cachedepth]}
    node[left] at (CACHE.bpin 12) {cache[\cachedepth]}
    node[left] at (CACHE.bpin 11) {valid\_out}
    node[left] at (CACHE.bpin 10) {addr\_out}
    node[left] at (CACHE.bpin 9) {out}
;
\draw (CACHE.bpin 7) -- ++(-\ps, 0)
    node[mux, anchor=brpin 1] (muxCACHE) {};
\draw
    (CACHE.bpin 3) -- (DECODER.bpin 53)
    (CACHE.bpin 4) -- (DECODER.bpin 52)

    % multiplexer
    (muxCACHE.btpin 1) |- (DECODER.bpin 51)
    (muxCACHE.blpin 2)
        -| ([shift={(\padding, 0)}] ST3.bpin 94 |- DECODER.bpin 59)
        -| ([shift={(\ps, 0)}] ALU.brpin 1) -- (ALU.brpin 1)
    (muxCACHE.blpin 1)
        -| ([shift={(\padding+\ps, 0)}] ST3.bpin 94 |- DECODER.bpin 58)
        -| ([shift={(2*\ps, 0)}] RAM.bpin 12 -| ALU.brpin 1) node[circ] (outRAM) {}
        -- (RAM.bpin 12)

    % cache buses
    (SEL2.bpin 7) -- ++(-\ps-0*\mp, 0) node[circ] (cache_in) {} |- (SEL1.bpin 7)
    (SEL2.bpin 6) -- ++(-\ps-1*\mp, 0) node[circ] (cache_addr) {} |- (SEL1.bpin 6)
    (SEL2.bpin 5) -- ++(-\ps-2*\mp, 0) node[circ] (cache_valid) {} |- (SEL1.bpin 5)
    (CACHE.bpin 12) -| ([shift={(\ps+2*\mp, -0*\mp)}] CACHE.bpin 12 |- DECODER.bpin 57) -| (cache_in)
    (CACHE.bpin 13) -| ([shift={(\ps+1*\mp, -1*\mp)}] CACHE.bpin 13 |- DECODER.bpin 57) -| (cache_addr)
    (CACHE.bpin 14) -| ([shift={(\ps+0*\mp, -2*\mp)}] CACHE.bpin 14 |- DECODER.bpin 57) -| (cache_valid)

    % return to REGISTER FILE
    (CACHE.bpin 11) -- ++(\ps+2*\mp, 0)
        |- ([shift={(\ps+\padding-2*\mp,-\padding-2*\mp)}] ST2.bpin 48)
        |- (RF.bpin 5)
    (CACHE.bpin 10) -- ++(\ps+1*\mp, 0)
        |- ([shift={(\ps+\padding-1*\mp,-\padding-1*\mp)}] ST2.bpin 48)
        |- (RF.bpin 6)
    (CACHE.bpin 9) -- ++(\ps+0*\mp, 0)
        |- ([shift={(\ps+\padding-0*\mp,-\padding-0*\mp)}] ST2.bpin 48)
        |- (RF.bpin 7)
;

% ST4
\draw ([shift={(\padding + 2*\ps, 0)}] ALU.brpin 1 |- ST1.bpin 1)
    node[stage, anchor=bpin 1, num pins=94] (ST4) {}
    node[below] at (ST4.n) {\normalsize 4}
    node[\namescolor] at ([shift={(0, \namesspacing)}] $(ST3.n)!0.5!(ST4.n)$) {\normalsize EXECUTE, MEMORY ACCESS}
;

% IP
\draw ([shift={(\padding+\ps, 0)}] ST4.bpin 86)
    node[mux, anchor=w] (muxIP) {};
\draw (muxIP.brpin 1) -- ++(\ps, 0)
    node[dipchip, anchor=bpin 9, num pins=22] (IP) {}
    node[below] at (IP.n) {\normalsize INDEX MANAGER}
    node[right] at (IP.bpin 3) {last\_index}
    node[right] at (IP.bpin 4) {jump}
    node[right] at (IP.bpin 5) {conditions}
    node[right] at (IP.bpin 6) {fallback\_index}
    node[right] at (IP.bpin 9) {jump\_index}
    node[right] at (IP.bpin 11) {flags}
    node[left] at (IP.bpin 20) {index}
;
\draw
    (muxIP.btpin 1) |- (DECODER.bpin 88)
    (muxIP.blpin 1) -- ++(-\ps, 0) |- (DECODER.bpin 87)
    (muxIP.blpin 2) -- ++(-\ps, 0) |- (outRAM)
    (IP.bpin 3) -| ++(-\ps, 2*\ps + \padding) node[circ] (index2) {} -| (index1)
    (IP.bpin 4) -- (DECODER.bpin 91)
    (IP.bpin 5) -- (DECODER.bpin 90)
    (IP.bpin 6) -- (DECODER.bpin 89)
    (IP.bpin 11) -- ++(-\ps, 0) |- (ALU.bbpin 2)
;

% ST5
\draw (IP.bpin 22) ++(\padding, 0)
    node[stage, anchor=bpin 1, num pins=94] (ST5) {}
    node[below] at (ST5.n) {\normalsize 5}
    node[\namescolor] at ([shift={(0, \namesspacing)}] $(ST4.n)!0.5!(ST5.n)$) {\normalsize FLOW CONTROL};
;
\draw
    (IP.bpin 20) -- (ST5.bpin 92) -- ++(\padding, 0) |- (index2)
;

% dashed lines
\foreach \i in {1,2,...,5} {
    \draw[thick, dashed, \namescolor]
        (ST\i.n) -- ++(0, \namesspacing+0.3)
        (ST\i.s) -- ++(0, -\namesspacing)
    ;
};
\end{circuitikz}
\end{document}
